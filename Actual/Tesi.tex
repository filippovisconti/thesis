 % Classe appositamente creata per tesi di Ingegneria Informatica all'università Roma Tre
\documentclass{TesiDiaUniroma3}

% --- INIZIO dati relativi al template TesiDiaUniroma3
% dati obbligatori, necessari al frontespizio
\titolo{Realizzazione di un sistema di cyber-defense: utilizzo delle VPN per un accesso remoto sicuro a risorse interne }
\autore{Filippo Visconti}
\matricola{547344}
\relatore{Prof. Maurizio Patrignani}
\correlatore{Federico Lommi} % modifica anche TesiDiaUniroma3.cls se vuoi avere un correlatore
\annoAccademico{2021/2022}

% dati opzionali
\dedica{Questa è la dedica} % solo se nel documento si usa il comando \generaDedica
% --- FINE dati relativi al template TesiDiaUniroma3

% --- INIZIO richiamo di pacchetti di utilità. Questi sono un esempio e non sono strettamente necessari al modello per la tesi.
\usepackage[plainpages=false]{hyperref}	% generazione di collegamenti ipertestuali su indice e riferimenti
\usepackage[all]{hypcap} % per far si che i link ipertestuali alle immagini puntino all'inizio delle stesse e non alla didascalia sottostante
\usepackage{amsthm}	% per definizioni e teoremi
\usepackage{amsmath}	% per ``cases'' environment
% --- FINE riachiamo di pacchetti di utilità

\begin{document}
% ----- Pagine di fronespizio, numerate in romano (i,ii,iii,iv...) (obbligatorio)
\frontmatter
\generaFrontespizio
\generaDedica
\ringraziamenti{ringraziamenti}	% inserisce i ringraziamenti e li prende in questo caso da ringraziamenti.tex
\introduzione{introduzione}		% inserisce l'introduzione e la prende in questo caso da introduzione.tex
\generaIndice
\generaIndiceFigure

% ----- Pagine di tesi, numerate in arabo (1,2,3,4,...) (obbligatorio)
\mainmatter
% il comando ``capitolo'' ha come parametri:
% 1) il titolo del capitolo
% 2) il nome del file tex (senza estensione) che contiene il capitolo. Tale nome \`e usato anche come label del capitolo
\capitolo{Requisiti}{capitolo-1}
\capitolo{Stato dell'arte}{capitolo-2}
\capitolo{Realizzazione}{capitolo-3}
\capitolo{Testing}{capitolo-4}
\capitolo{Security concerns}{capitolo-5}

% ----- Parte finale della tesi (obbligatorio)
\backmatter
\conclusioni{conclusioni}

% Bibliografia con BibTeX (obbligatoria)
% Non si deve specificare lo stile della bibliografia
\bibliography{bibliografia} % inserisce la bibliografia e la prende in questo caso da bibliografia.bib

\end{document}