%--------------------------------------------------------------------------------------------------------------------------------------------VPN
\section{Virtual Private Networks }

\subsection{Concetti fondamentali}
Ancora del testo

\subsection{Architetture disponibili}
Ancora del testo

\subsubsection{Gateway-to-Gateway}
VPN gateway may be a dedicated de- vice or just a part of network device, such a router or firewall [11]. VPN
3 Background 11
gateway essentially separates internal network from anything over other side of gateway.


\subsubsection{Host-to-Host}
Primarily used to provide secure remote access. The host (IPsec client) uses VPN tunnel to connect to a VPN gateway. Whenever a host wishes to create a VPN connection to a server, he must authenticate and establish a connection with a gateway, which then manages host's connection to a VPN server.

\subsubsection{Host-to-Gateway}
Quella remote access

\subsection{Soluzioni principali}
PPTP, IPSec, OpenVPN, WireGuard.

\subsection{Perché soddisfano i requisiti}
Prova di testo di capitolo. Vorrei citare qui tutta l'opera omnia di~\cite{IEEE:1990,WIKI:INTEROP,BOX:1997,AHL:1996}.
%---------------------------------------------------------------------------------------------------------------------------------------------IPSec
% BUONO https://pctempo.com/openvpn-vs-ikev2-vs-pptp-vs-l2tp-ipsec-vs-sstp/

% https://it.wikipedia.org/wiki/IPsec https://github.com/sunknudsen/privacy-guides/blob/master/how-to-self-host-hardened-strongswan-ikev2-ipsec-vpn-server-for-ios-and-macos/README.md 

% 1.	Come è nato
% 2.	Tipo incapsulamento
% 3.	Overhead - byte sprecati per pacchetto
% 4.	Livello a cui lavora
% 5.	Protocolli usati
% 6.	Cifratura usata ? hw o sw, limitata se non aggiorni hw ma più veloce, e il contrario

\section{IPSec}
\subsection{Panoramica}
IPsec is a framework of open standards for ensuring private communications over IP networks which has become the most commonly used network layer security control [11]. IPsec is based on securing Network layer of TCP/IP model. In many environments securing Network layer is a better solution than securing higher Transport or Application layers. It makes a way for network administrators to enforce certain security policies, and also provides a more flexible way in protecting IP information for each packet [11]. Depending on the implementation IPsec can provide a combination of following security measures: confidentiality, integrity, peer authentication, replay protection, traffic analysis protection and access control
\subsection{Protocolli utilizzati}
Ancora del testo

\subsubsection{Authentication Header}
Ancora del testo

\subsubsection{Encapsulating Security Payload}
Ancora del testo

\subsubsection{Internet Key Exchange v2}
Ancora del testo---IKE è un acronimo per Internet key exchange ed è il protocollo usato per stabilire una security association nella suite di protocolli IPsec. Questo protocollo è definito in RFC 4306. È un protocollo di livello applicazione e utilizza il protocollo UDP come protocollo di trasporto; la porta su cui viene stabilita la connessione è 500. ---

\subsection{Transport mode vs Tunnel mode}
Ancora del testo

\subsection{Cifratura}
Ancora del testo

\subsection{Autenticazione}
Ancora del testo

\subsection{Implementazioni}
StrongSwan is an open source IPsec implementation for the Linux oper- ating system [18]. Maintained by Andreas Steffen, strongSwan supports features, such as IPv6, Android 4+, X.509 public key certificates, certifi- cate revocation lists, RSA private key storage on smartcards, ability to interoperate with various MS Windows and Mac OS X VPN clients, full implementation of IKEv2 protocol, and much more.

\subsection{Considerazioni}
Ancora del testo
%---------------------------------------------------------------------------------------------------------------------------------------------PPTP
% https://www.ivpn.net/pptp-vs-ipsec-ikev2-vs-openvpn-vs-wireguard/

\section{PPTP}
\subsection{Panoramica}
Point-to-Point Tunneling Protocol (PPTP) is a virtual private network imple- mentation method which uses TCP control channel and a Generic Routing En- capsulation (GRE) tunnel to encapsulate Point-to-Point Protocol (PPP) [16] packets and send them over TCP/IP links. The protocol was developed by a vendor consortium and documented in RFC 2637 [17]. PPTP encapsulated virtual network packets inside the PPP packets, which are then encapsulated

3 Background 13
inside the GRE packets and these encapsulated inside the TCP control channel. Everything is then sent over IP network on TCP port 1723.
\subsection{Protocolli utilizzati}
Ancora del testo

\subsubsection{Something}
Ancora del testo

\subsection{TCP vs UDP}
Ancora del testo

\subsection{Cifratura}
NESSUNA

\subsection{Autenticazione}
Ancora del testo

\subsection{Considerazioni}
Ancora del testo

%----------------------------------------------------------------------------------------------------------------------------------------------OpenVPN
% https://en.wikipedia.org/wiki/OpenVPN 
\section{OpenVPN}
\subsection{Panoramica}
OpenVPN is an SSL VPN implementation which implements OSI layer 2 or 3 secure network extension using the industry standard TLS proto- col [19].

\subsection{Protocolli utilizzati}
Ancora del testo

\subsubsection{Something}
Ancora del testo

\subsection{TCP vs UDP}
Ancora del testo

\subsection{Cifratura}
Ancora del testo

\subsection{Autenticazione}
Ancora del testo

\subsection{Misure di sicurezza aggiuntive}
Ancora del testo
OpenVPN offers various internal security features. It has up to 256-bit encryption through the OpenSSL library, although some service providers may offer lower rates, effectively providing some of the fastest VPN available to consumers. It runs in userspace instead of requiring IP stack (therefore kernel) operation. OpenVPN has the ability to drop root privileges, use mlockall to prevent swapping sensitive data to disk, enter a chroot jail after initialization, and apply a SELinux context after initialization.

OpenVPN runs a custom security protocol based on SSL and TLS, rather than supporting IKE, IPsec, L2TP or PPTP.

OpenVPN offers support of smart cards via PKCS 11-based cryptographic tokens.
\subsection{Considerazioni}
Ancora del testo

%--------------------------------------------------------------------------------------------------------------------------------------------Wireguard
\section{WireGuard}
\subsection{Panoramica}
Prova di testo di capitolo. Vorrei citare qui tutta l'opera omnia di~\cite{IEEE:1990,WIKI:INTEROP,BOX:1997,AHL:1996}.

\subsection{Protocolli utilizzati}
Ancora del testo

\subsubsection{Something}
Ancora del testo

\subsection{TCP vs UDP}
Ancora del testo

\subsection{Cifratura}
Ancora del testo

\subsection{Autenticazione}
Ancora del testo

\subsection{Considerazioni}
Ancora del testo