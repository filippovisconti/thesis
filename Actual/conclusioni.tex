\section{Quale è uscito vincitore}
Ancora del testo

\section{Come migliorare le misure}
Ancora del testo

\section{Test di VPN Peer-To-Peer}
Nel panorama delle soluzioni VPN disponibili, esiste una ulteriore tipologia che sfrutta la rete Peer-To-Peer, su cui instaura una overlay network virtuale e privata.

\subsection{Zero Tier}
Si tratta di un hypervisor di rete distrubuito, costruito sopra una rete globale Peer-To-Peer crittograficamente sicura.

L'hypervisor di rete di ZeroTier è un virtualizzatore di reti autocontenuto che implementa un livello 2 virtuale al di sopra di una rete Peer-To-Peer cifrata globale.

Il protocollo utilizzato da ZeroTier è originale, anche se alcuni suoi aspetti sono simili a VXLAN e IPSec. Consiste in due livelli concettualmente separati ma fortemente accoppiati: VL1 e VL2, che corrispondono ai livelli 1 e 2 del modello OSI. VL1 è formato dal livello di trasporto Peer-To-Peer, il "cavo virtuale"; VL2 è un livello 2 emulato che offre ai sistemi operativi e agli applicativi un mezzo di comunicazione familiare.

Nelle reti tradizionali, il livello 1 del modello OSI rappresenta il cavo fisico virtuale o il canale radio di comunicazione attraverso il quale i dati sono trasmessi. VL1 svolge lo stesso compito, utilizzando cifratura e autenticazione, oltre ad un'altra serie di artifici per creare cavi virtuali al bisogno, in maniera dinamica.

Per raggiungere questo scopo, VL1 è organizzato con uno schema simile al DNS. Alla base della rete c'è una collezione di root server sempre attivi, il cui ruolo è simile a quello dei DNS root name server. Sui root server gira lo stesso software degli endpoint, ma la loro posizione è fissa e nota, e hanno performance elevatissime.

% -------

First off, you can compare them like dropbox and hosting your own FTP server. Sure, before dropbox came along, people thought there wasn't any use for dropbox since people could host their own FTP servers and access them from anywhere. Now, people who know how to setup an FTP server use dropbox for the convenience.
I'll start with the similarities:
Both of them can be used for in-house streaming, VLAN for gaming, etc.
Both of them are secure and you don't have to worry about transport security.
Both of them introduce overhead to your network but in different levels, depending on your situation.
Now, the dissimilarities:
The most important difference is that ZeroTier is supposed to be a peer-to-peer connectivity system. It does this by doing something called UDP hole punching. Which is basically tricking the router into letting someone access a port on the computer directly without TCP connection establishment. But OpenVPN routes ALL of the traffic meant for a client on the same network through the server. This usually results in better speed and bandwidth savings for your server in the cloud, because in Zerotier, two clients directly communicate with each other and that is one less server to traverse through.
Time to setup. It takes less than 2 minutes to set-up Zerotier unless you aren't experienced with networking stuff, in which case the the zerotier console setting would take a minute more. OpenVPN, is a bit of a pain, and you can easily make a mistake.
Getting around network restrictions. You'll have to figure out by yourself how to get across your network if there are strict rules. You'll have to manually decide what port to use, TCP or UDP, etc. But zerotier tests different ports, starting with 9993/udp then eventually to 443/tcp which most networks should let you do. (This won't work of you have a whitelisted firewall or have something like BlueCoat)
Free! For openvpn you'll have to set up your own server, and configure, maintain, etc. You WILL have to pay for this, unless you have a static public IP. But zerotier needs no managing and will be cheaper, since for the connection establishment, the root servers are used which are managed by zerotier.
I could go into performance, latency, etc. But that will mostly depend of location, your network's restrictiveness, and other things. OpenVPN could be more performant and vice-versa.
And you would not really have a need to use them together. Let me know if I need to explain anything further. A simple way to decide would be:
Are you okay with setting up a openvpn server by yourself and do everything needed with ALL of your devices? If yes, then OpenVPN.
Do you have a HARD latency requirement? Then, you'll have to do some testing to figure out which will be faster.



\section{PAM vs VPN}
Una soluzione alternativa alle VPN per garantire un accesso remoto sicuro a risorse interne è un meccanismo chiamato Privileged Access Management (PAM).
Per minimizzare i rischi correlati all'accesso con VPN, è opportuno che ogni client che si colleghi abbia accesso solo ed esclusivamente ai sistemi di cui effettivamente necessitano per portare a termine il loro lavoro con successo.
Purtroppo, questo controllo a grana fine non è raggiungibile in maniera efficente utilizzando soltanto una soluzione VPN. In soccorso arrivano soluzioni per gestire gli accessi privilegiati, quali PAM - privileged access management.
PAM permette alle aziende di dare ai dipendenti o ai clienti accesso alla propria rete interna senza una connessione VPN e permette allo staff IT di controllare, monitorare e gestire l'accesso a risorse critiche. Questo consente alle aziende di sapere con precisione quali account sono responsibili di quali attività.
Introducendo diversi livelli di privilegi, PAM riduce anche la superficie esposta ad attacchi.

% https://www.gb-advisors.com/vpn-vs-pam-which-remote-access-technologies-more-secure-business/